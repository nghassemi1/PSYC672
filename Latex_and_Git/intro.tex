\section{Introduction}

According to the National Survey of Children’s Health, researchers estimate that more than five million children experienced parental incarceration in 2015 \citep{murphey2015parents}. This study also shows that the majority of these children fall within the school age range and are disproportionately Black. Parental incarceration is identified as an adverse childhood experience (ACE) that often co-occurs with other ACEs such as living with a household drug user or witnessing abuse \citep{scott2013interrelation}. This co-occurrence can intensify the negative impacts of parental incarceration (e.g., increased risk of intergenerational crime, substance use, and depressive symptoms) on children \citep{poehlmann2019research}. 

Parental incarceration substantially impacts family units, especially the children within these families. Children of incarcerated parents (COIP) often develop insecure attachments with their caregivers \citep{murray2010parental}, which contributes to a range of emotional and psychological difficulties, including depression \citep{spruit2020relation}. Furthermore, a study by \citet{dallaire2010relation} revealed how children who witnessed parental incarceration-related events can further impact them as they showed more depressive and anxiety symptoms than COIP who did not witness such events. This study also found that the children who witnessed parental incarceration-related events performed worse academically than those who did not.

A wide body of research demonstrates the significant roles that emotion regulation and resilience play in a child's overall emotional well-being \citep{macklem2010importance, mihic2018importance, morie2022process}. These constructs are especially important for COIP, as they are at an increased risk of exposure to adversity.

The current research examines changes in children's emotional competencies as they progress through a SEL program designed for COIP. We are particularly interested in the long-term impacts of the program and determining how the effects of the program may differ for returners to the program compared to new members. Overall, we expect emotion coping scores to increase from fall to spring each academic year (hypothesis one) and inhibited and dysregulated emotional responses to decrease (hypothesis two). We also expect resilience scores to increase overall (hypothesis three). 