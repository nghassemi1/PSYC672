\section{Method}
This study was conducted in collaboration with the non-profit Assisting Families of Inmates (AFOI) and Virginia Commonwealth University (VCU). AFOI manages participant recruitment and trains the VCU Master of Social Work (MSW) interns who administer the program and lead each session. Through the MAC program, AFOI is dedicated to supporting elementary-aged school children with one or both parents, or other family members, incarcerated. 

\subsection{Participants}

Participants were recruited from eight public schools across Richmond, Virginia. They must have had at least one incarcerated parent or family member and be enrolled in the MAC program at their school. At the beginning of each school year, teachers distributed an interest form for parents and caregivers to complete to enroll their child in the program.

Participants' ages ranged from seven to 11 years (grades second through fifth), and all participants were African American or Black, except for one who was Hispanic. 

\subsection{Procedure}

Trained research assistants traveled to public schools for data collection. Baseline data was collected in the early fall of each academic year, a few weeks after the start of school, and follow-up data collection occurred in the spring of each academic year, about six months later. Makeup data collection days were scheduled for the following week of each data collection time point for children who had been absent or enrolled later. Each caregiver gave consent for their child to participate, and child participants also gave their assent before completing the surveys.
Qualtrics surveys were administered digitally via tablets. Through the surveys, participants complete measures of rugged resilience and anger and sadness management in the early fall and spring of each academic year. Research assistants read surveys to children unable to do so themselves and clarified any confusing questions or vocabulary. Once the survey was complete, responses were uploaded into Qualtrics, and children were compensated for their time with pencils and sticker sheets.